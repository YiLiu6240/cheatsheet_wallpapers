\documentclass[]{article}
\usepackage{lmodern}
\usepackage{amssymb,amsmath}
\usepackage{ifxetex,ifluatex}
\usepackage{fixltx2e} % provides \textsubscript
\ifnum 0\ifxetex 1\fi\ifluatex 1\fi=0 % if pdftex
  \usepackage[T1]{fontenc}
  \usepackage[utf8]{inputenc}
\else % if luatex or xelatex
  \ifxetex
    \usepackage{mathspec}
  \else
    \usepackage{fontspec}
  \fi
  \defaultfontfeatures{Ligatures=TeX,Scale=MatchLowercase}
\fi
% use upquote if available, for straight quotes in verbatim environments
\IfFileExists{upquote.sty}{\usepackage{upquote}}{}
% use microtype if available
\IfFileExists{microtype.sty}{%
\usepackage{microtype}
\UseMicrotypeSet[protrusion]{basicmath} % disable protrusion for tt fonts
}{}
\usepackage[unicode=true]{hyperref}
\hypersetup{
            pdftitle={Comparison cheatsheet},
            pdfborder={0 0 0},
            breaklinks=true}
\urlstyle{same}  % don't use monospace font for urls
\usepackage{listings}
\usepackage{longtable,booktabs}
\IfFileExists{parskip.sty}{%
\usepackage{parskip}
}{% else
\setlength{\parindent}{0pt}
\setlength{\parskip}{6pt plus 2pt minus 1pt}
}
\setlength{\emergencystretch}{3em}  % prevent overfull lines
\providecommand{\tightlist}{%
  \setlength{\itemsep}{0pt}\setlength{\parskip}{0pt}}
\setcounter{secnumdepth}{0}
% Redefines (sub)paragraphs to behave more like sections
\ifx\paragraph\undefined\else
\let\oldparagraph\paragraph
\renewcommand{\paragraph}[1]{\oldparagraph{#1}\mbox{}}
\fi
\ifx\subparagraph\undefined\else
\let\oldsubparagraph\subparagraph
\renewcommand{\subparagraph}[1]{\oldsubparagraph{#1}\mbox{}}
\fi

\title{Comparison cheatsheet}
\date{}

\begin{document}
\maketitle

\section{Creating Vectors}\label{creating-vectors}

\begin{longtable}[]{@{}llll@{}}
\toprule
\begin{minipage}[b]{0.21\columnwidth}\raggedright\strut
Operation\strut
\end{minipage} & \begin{minipage}[b]{0.19\columnwidth}\raggedright\strut
MATLAB\strut
\end{minipage} & \begin{minipage}[b]{0.29\columnwidth}\raggedright\strut
Python\strut
\end{minipage} & \begin{minipage}[b]{0.19\columnwidth}\raggedright\strut
Julia\strut
\end{minipage}\tabularnewline
\midrule
\endhead
\begin{minipage}[t]{0.21\columnwidth}\raggedright\strut
Row vector: size (1, n)\strut
\end{minipage} & \begin{minipage}[t]{0.19\columnwidth}\raggedright\strut
\begin{lstlisting}[language=Matlab]
A = [1 2 3]
\end{lstlisting}
\strut
\end{minipage} & \begin{minipage}[t]{0.29\columnwidth}\raggedright\strut
\begin{lstlisting}[language=Python]
A = np.array([1, 2, 3]).reshape(1,3)
\end{lstlisting}
\strut
\end{minipage} & \begin{minipage}[t]{0.19\columnwidth}\raggedright\strut
\begin{lstlisting}
A = [1 2 3]
\end{lstlisting}
\strut
\end{minipage}\tabularnewline
\begin{minipage}[t]{0.21\columnwidth}\raggedright\strut
Column vector: size (n, 1)\strut
\end{minipage} & \begin{minipage}[t]{0.19\columnwidth}\raggedright\strut
\begin{lstlisting}[language=Matlab]
A = [1; 2; 3]
\end{lstlisting}
\strut
\end{minipage} & \begin{minipage}[t]{0.29\columnwidth}\raggedright\strut
\begin{lstlisting}[language=Python]
A = np.array([1, 2, 3]).reshape(3,1)
\end{lstlisting}
\strut
\end{minipage} & \begin{minipage}[t]{0.19\columnwidth}\raggedright\strut
\begin{lstlisting}
A = [1 2 3]'
\end{lstlisting}
\strut
\end{minipage}\tabularnewline
\begin{minipage}[t]{0.21\columnwidth}\raggedright\strut
1d array: size (n, )\strut
\end{minipage} & \begin{minipage}[t]{0.19\columnwidth}\raggedright\strut
Not possible\strut
\end{minipage} & \begin{minipage}[t]{0.29\columnwidth}\raggedright\strut
\begin{lstlisting}[language=Python]
A = np.array([1, 2, 3])
\end{lstlisting}
\strut
\end{minipage} & \begin{minipage}[t]{0.19\columnwidth}\raggedright\strut
\begin{lstlisting}
A = [1; 2; 3]
\end{lstlisting}

or

\begin{lstlisting}
A = [1, 2, 3]
\end{lstlisting}
\strut
\end{minipage}\tabularnewline
\begin{minipage}[t]{0.21\columnwidth}\raggedright\strut
Integers from j to n with step size k\strut
\end{minipage} & \begin{minipage}[t]{0.19\columnwidth}\raggedright\strut
\begin{lstlisting}[language=Matlab]
A = j:k:n
\end{lstlisting}
\strut
\end{minipage} & \begin{minipage}[t]{0.29\columnwidth}\raggedright\strut
\begin{lstlisting}[language=Python]
A = np.arange(j, n+1, k)
\end{lstlisting}
\strut
\end{minipage} & \begin{minipage}[t]{0.19\columnwidth}\raggedright\strut
\begin{lstlisting}
A = j:k:n
\end{lstlisting}
\strut
\end{minipage}\tabularnewline
\begin{minipage}[t]{0.21\columnwidth}\raggedright\strut
Linearly spaced vector of k points\strut
\end{minipage} & \begin{minipage}[t]{0.19\columnwidth}\raggedright\strut
\begin{lstlisting}[language=Matlab]
A = linspace(1, 5, k)
\end{lstlisting}
\strut
\end{minipage} & \begin{minipage}[t]{0.29\columnwidth}\raggedright\strut
\begin{lstlisting}[language=Python]
A = np.linspace(1, 5, k)
\end{lstlisting}
\strut
\end{minipage} & \begin{minipage}[t]{0.19\columnwidth}\raggedright\strut
\begin{lstlisting}
A = linspace(1, 5, k)
\end{lstlisting}
\strut
\end{minipage}\tabularnewline
\bottomrule
\end{longtable}

\section{Creating Matrices}\label{creating-matrices}

\begin{longtable}[]{@{}llll@{}}
\toprule
\begin{minipage}[b]{0.24\columnwidth}\raggedright\strut
Operation\strut
\end{minipage} & \begin{minipage}[b]{0.20\columnwidth}\raggedright\strut
MATLAB\strut
\end{minipage} & \begin{minipage}[b]{0.25\columnwidth}\raggedright\strut
Python\strut
\end{minipage} & \begin{minipage}[b]{0.20\columnwidth}\raggedright\strut
Julia\strut
\end{minipage}\tabularnewline
\midrule
\endhead
\begin{minipage}[t]{0.24\columnwidth}\raggedright\strut
Create a matrix\strut
\end{minipage} & \begin{minipage}[t]{0.20\columnwidth}\raggedright\strut
\begin{lstlisting}[language=Matlab]
A = [1 2; 3 4]
\end{lstlisting}
\strut
\end{minipage} & \begin{minipage}[t]{0.25\columnwidth}\raggedright\strut
\begin{lstlisting}[language=Python]
A = np.array([[1, 2], [3, 4]])
\end{lstlisting}
\strut
\end{minipage} & \begin{minipage}[t]{0.20\columnwidth}\raggedright\strut
\begin{lstlisting}
A = [1 2; 3 4]
\end{lstlisting}
\strut
\end{minipage}\tabularnewline
\begin{minipage}[t]{0.24\columnwidth}\raggedright\strut
2 x 2 matrix of zeros\strut
\end{minipage} & \begin{minipage}[t]{0.20\columnwidth}\raggedright\strut
\begin{lstlisting}[language=Matlab]
A = zeros(2, 2)
\end{lstlisting}
\strut
\end{minipage} & \begin{minipage}[t]{0.25\columnwidth}\raggedright\strut
\begin{lstlisting}[language=Python]
A = np.zeros((2, 2))
\end{lstlisting}
\strut
\end{minipage} & \begin{minipage}[t]{0.20\columnwidth}\raggedright\strut
\begin{lstlisting}
A = zeros(2, 2)
\end{lstlisting}
\strut
\end{minipage}\tabularnewline
\begin{minipage}[t]{0.24\columnwidth}\raggedright\strut
2 x 2 matrix of ones\strut
\end{minipage} & \begin{minipage}[t]{0.20\columnwidth}\raggedright\strut
\begin{lstlisting}[language=Matlab]
A = ones(2, 2)
\end{lstlisting}
\strut
\end{minipage} & \begin{minipage}[t]{0.25\columnwidth}\raggedright\strut
\begin{lstlisting}[language=Python]
A = np.ones((2, 2))
\end{lstlisting}
\strut
\end{minipage} & \begin{minipage}[t]{0.20\columnwidth}\raggedright\strut
\begin{lstlisting}
A = ones(2, 2)
\end{lstlisting}
\strut
\end{minipage}\tabularnewline
\begin{minipage}[t]{0.24\columnwidth}\raggedright\strut
2 x 2 identity matrix\strut
\end{minipage} & \begin{minipage}[t]{0.20\columnwidth}\raggedright\strut
\begin{lstlisting}[language=Matlab]
A = eye(2, 2)
\end{lstlisting}
\strut
\end{minipage} & \begin{minipage}[t]{0.25\columnwidth}\raggedright\strut
\begin{lstlisting}[language=Python]
A = np.eye(2)
\end{lstlisting}
\strut
\end{minipage} & \begin{minipage}[t]{0.20\columnwidth}\raggedright\strut
\begin{lstlisting}
A = eye(2, 2)
\end{lstlisting}
\strut
\end{minipage}\tabularnewline
\begin{minipage}[t]{0.24\columnwidth}\raggedright\strut
Diagonal matrix\strut
\end{minipage} & \begin{minipage}[t]{0.20\columnwidth}\raggedright\strut
\begin{lstlisting}[language=Matlab]
A = diag([1 2 3])
\end{lstlisting}
\strut
\end{minipage} & \begin{minipage}[t]{0.25\columnwidth}\raggedright\strut
\begin{lstlisting}[language=Python]
A = np.diag([1, 2, 3])
\end{lstlisting}
\strut
\end{minipage} & \begin{minipage}[t]{0.20\columnwidth}\raggedright\strut
\begin{lstlisting}
A = diagm([1; 2; 3])
\end{lstlisting}
\strut
\end{minipage}\tabularnewline
\begin{minipage}[t]{0.24\columnwidth}\raggedright\strut
Uniform random numbers\strut
\end{minipage} & \begin{minipage}[t]{0.20\columnwidth}\raggedright\strut
\begin{lstlisting}[language=Matlab]
A = rand(2, 2)
\end{lstlisting}
\strut
\end{minipage} & \begin{minipage}[t]{0.25\columnwidth}\raggedright\strut
\begin{lstlisting}[language=Python]
A = np.random.rand(2,2)
\end{lstlisting}
\strut
\end{minipage} & \begin{minipage}[t]{0.20\columnwidth}\raggedright\strut
\begin{lstlisting}
A = rand(2, 2)
\end{lstlisting}
\strut
\end{minipage}\tabularnewline
\begin{minipage}[t]{0.24\columnwidth}\raggedright\strut
Normal random numbers\strut
\end{minipage} & \begin{minipage}[t]{0.20\columnwidth}\raggedright\strut
\begin{lstlisting}[language=Matlab]
A = randn(2, 2)
\end{lstlisting}
\strut
\end{minipage} & \begin{minipage}[t]{0.25\columnwidth}\raggedright\strut
\begin{lstlisting}[language=Python]
A = np.random.randn(2, 2)
\end{lstlisting}
\strut
\end{minipage} & \begin{minipage}[t]{0.20\columnwidth}\raggedright\strut
\begin{lstlisting}
A = randn(2, 2)
\end{lstlisting}
\strut
\end{minipage}\tabularnewline
\bottomrule
\end{longtable}

\section{Manipulating Vectors and
Matrices}\label{manipulating-vectors-and-matrices}

\begin{longtable}[]{@{}llll@{}}
\toprule
\begin{minipage}[b]{0.24\columnwidth}\raggedright\strut
Operation\strut
\end{minipage} & \begin{minipage}[b]{0.23\columnwidth}\raggedright\strut
MATLAB\strut
\end{minipage} & \begin{minipage}[b]{0.20\columnwidth}\raggedright\strut
Python\strut
\end{minipage} & \begin{minipage}[b]{0.20\columnwidth}\raggedright\strut
Julia\strut
\end{minipage}\tabularnewline
\midrule
\endhead
\begin{minipage}[t]{0.24\columnwidth}\raggedright\strut
Transpose\strut
\end{minipage} & \begin{minipage}[t]{0.23\columnwidth}\raggedright\strut
\begin{lstlisting}[language=Matlab]
A.'
\end{lstlisting}
\strut
\end{minipage} & \begin{minipage}[t]{0.20\columnwidth}\raggedright\strut
\begin{lstlisting}[language=Python]
A.T
\end{lstlisting}
\strut
\end{minipage} & \begin{minipage}[t]{0.20\columnwidth}\raggedright\strut
\begin{lstlisting}
A.'
\end{lstlisting}
\strut
\end{minipage}\tabularnewline
\begin{minipage}[t]{0.24\columnwidth}\raggedright\strut
Complex conjugate transpose\strut
\end{minipage} & \begin{minipage}[t]{0.23\columnwidth}\raggedright\strut
\begin{lstlisting}[language=Matlab]
A'
\end{lstlisting}
\strut
\end{minipage} & \begin{minipage}[t]{0.20\columnwidth}\raggedright\strut
\begin{lstlisting}[language=Python]
A.conj()
\end{lstlisting}
\strut
\end{minipage} & \begin{minipage}[t]{0.20\columnwidth}\raggedright\strut
\begin{lstlisting}
A'
\end{lstlisting}
\strut
\end{minipage}\tabularnewline
\begin{minipage}[t]{0.24\columnwidth}\raggedright\strut
Concatenate horizontally\strut
\end{minipage} & \begin{minipage}[t]{0.23\columnwidth}\raggedright\strut
\begin{lstlisting}[language=Matlab]
A = [[1 2] [1 2]]
\end{lstlisting}

or

\begin{lstlisting}[language=Matlab]
A = horzcat([1 2], [1 2])
\end{lstlisting}
\strut
\end{minipage} & \begin{minipage}[t]{0.20\columnwidth}\raggedright\strut
\begin{lstlisting}[language=Python]
B = np.array([1, 2])
A = np.hstack((B, B))
\end{lstlisting}
\strut
\end{minipage} & \begin{minipage}[t]{0.20\columnwidth}\raggedright\strut
\begin{lstlisting}
A = [[1 2] [1 2]]
\end{lstlisting}

or

\begin{lstlisting}
A = hcat([1 2], [1 2])
\end{lstlisting}
\strut
\end{minipage}\tabularnewline
\begin{minipage}[t]{0.24\columnwidth}\raggedright\strut
Concatenate vertically\strut
\end{minipage} & \begin{minipage}[t]{0.23\columnwidth}\raggedright\strut
\begin{lstlisting}[language=Matlab]
A = [[1 2]; [1 2]]
\end{lstlisting}

or

\begin{lstlisting}[language=Matlab]
A = vertcat([1 2], [1 2])
\end{lstlisting}
\strut
\end{minipage} & \begin{minipage}[t]{0.20\columnwidth}\raggedright\strut
\begin{lstlisting}[language=Python]
B = np.array([1, 2])
A = np.vstack((B, B))
\end{lstlisting}
\strut
\end{minipage} & \begin{minipage}[t]{0.20\columnwidth}\raggedright\strut
\begin{lstlisting}
A = [[1 2]; [1 2]]
\end{lstlisting}

or

\begin{lstlisting}
A = vcat([1 2], [1 2])
\end{lstlisting}
\strut
\end{minipage}\tabularnewline
\begin{minipage}[t]{0.24\columnwidth}\raggedright\strut
Reshape (to 5 rows, 2 columns)\strut
\end{minipage} & \begin{minipage}[t]{0.23\columnwidth}\raggedright\strut
\begin{lstlisting}[language=Matlab]
A = reshape(1:10, 5, 2)
\end{lstlisting}
\strut
\end{minipage} & \begin{minipage}[t]{0.20\columnwidth}\raggedright\strut
\begin{lstlisting}[language=Python]
A = A.reshape(5,2)
\end{lstlisting}
\strut
\end{minipage} & \begin{minipage}[t]{0.20\columnwidth}\raggedright\strut
\begin{lstlisting}
A = reshape(1:10, 5, 2)
\end{lstlisting}
\strut
\end{minipage}\tabularnewline
\begin{minipage}[t]{0.24\columnwidth}\raggedright\strut
Convert matrix to vector\strut
\end{minipage} & \begin{minipage}[t]{0.23\columnwidth}\raggedright\strut
\begin{lstlisting}[language=Matlab]
A(:)
\end{lstlisting}
\strut
\end{minipage} & \begin{minipage}[t]{0.20\columnwidth}\raggedright\strut
\begin{lstlisting}[language=Python]
A = A.flatten()
\end{lstlisting}
\strut
\end{minipage} & \begin{minipage}[t]{0.20\columnwidth}\raggedright\strut
\begin{lstlisting}
A[:]
\end{lstlisting}
\strut
\end{minipage}\tabularnewline
\begin{minipage}[t]{0.24\columnwidth}\raggedright\strut
Flip left/right\strut
\end{minipage} & \begin{minipage}[t]{0.23\columnwidth}\raggedright\strut
\begin{lstlisting}[language=Matlab]
fliplr(A)
\end{lstlisting}
\strut
\end{minipage} & \begin{minipage}[t]{0.20\columnwidth}\raggedright\strut
\begin{lstlisting}[language=Python]
np.fliplr(A)
\end{lstlisting}
\strut
\end{minipage} & \begin{minipage}[t]{0.20\columnwidth}\raggedright\strut
\begin{lstlisting}
flipdim(A, 2)
\end{lstlisting}
\strut
\end{minipage}\tabularnewline
\begin{minipage}[t]{0.24\columnwidth}\raggedright\strut
Flip up/down\strut
\end{minipage} & \begin{minipage}[t]{0.23\columnwidth}\raggedright\strut
\begin{lstlisting}[language=Matlab]
flipud(A)
\end{lstlisting}
\strut
\end{minipage} & \begin{minipage}[t]{0.20\columnwidth}\raggedright\strut
\begin{lstlisting}[language=Python]
np.flipud(A)
\end{lstlisting}
\strut
\end{minipage} & \begin{minipage}[t]{0.20\columnwidth}\raggedright\strut
\begin{lstlisting}
flipdim(A, 1)
\end{lstlisting}
\strut
\end{minipage}\tabularnewline
\begin{minipage}[t]{0.24\columnwidth}\raggedright\strut
Repeat matrix (3 times in the row dimension, 4 times in the column
dimension)\strut
\end{minipage} & \begin{minipage}[t]{0.23\columnwidth}\raggedright\strut
\begin{lstlisting}[language=Matlab]
repmat(A, 3, 4)
\end{lstlisting}
\strut
\end{minipage} & \begin{minipage}[t]{0.20\columnwidth}\raggedright\strut
\begin{lstlisting}[language=Python]
np.tile(A, (4, 3))
\end{lstlisting}
\strut
\end{minipage} & \begin{minipage}[t]{0.20\columnwidth}\raggedright\strut
\begin{lstlisting}
repmat(A, 3, 4)
\end{lstlisting}
\strut
\end{minipage}\tabularnewline
\bottomrule
\end{longtable}

\section{Accessing Vector/Matrix
Elements}\label{accessing-vectormatrix-elements}

\begin{longtable}[]{@{}llll@{}}
\toprule
\begin{minipage}[b]{0.23\columnwidth}\raggedright\strut
Operation\strut
\end{minipage} & \begin{minipage}[b]{0.23\columnwidth}\raggedright\strut
MATLAB\strut
\end{minipage} & \begin{minipage}[b]{0.23\columnwidth}\raggedright\strut
Python\strut
\end{minipage} & \begin{minipage}[b]{0.20\columnwidth}\raggedright\strut
Julia\strut
\end{minipage}\tabularnewline
\midrule
\endhead
\begin{minipage}[t]{0.23\columnwidth}\raggedright\strut
Access one element\strut
\end{minipage} & \begin{minipage}[t]{0.23\columnwidth}\raggedright\strut
\begin{lstlisting}[language=Matlab]
A(2, 2)
\end{lstlisting}
\strut
\end{minipage} & \begin{minipage}[t]{0.23\columnwidth}\raggedright\strut
\begin{lstlisting}[language=Python]
A[1, 1]
\end{lstlisting}
\strut
\end{minipage} & \begin{minipage}[t]{0.20\columnwidth}\raggedright\strut
\begin{lstlisting}
A[2, 2]
\end{lstlisting}
\strut
\end{minipage}\tabularnewline
\begin{minipage}[t]{0.23\columnwidth}\raggedright\strut
Access specific rows\strut
\end{minipage} & \begin{minipage}[t]{0.23\columnwidth}\raggedright\strut
\begin{lstlisting}[language=Matlab]
A(1:4, :)
\end{lstlisting}
\strut
\end{minipage} & \begin{minipage}[t]{0.23\columnwidth}\raggedright\strut
\begin{lstlisting}[language=Python]
A[0:4, :]
\end{lstlisting}
\strut
\end{minipage} & \begin{minipage}[t]{0.20\columnwidth}\raggedright\strut
\begin{lstlisting}
A[1:4, :]
\end{lstlisting}
\strut
\end{minipage}\tabularnewline
\begin{minipage}[t]{0.23\columnwidth}\raggedright\strut
Access specific columns\strut
\end{minipage} & \begin{minipage}[t]{0.23\columnwidth}\raggedright\strut
\begin{lstlisting}[language=Matlab]
A(:, 1:4)
\end{lstlisting}
\strut
\end{minipage} & \begin{minipage}[t]{0.23\columnwidth}\raggedright\strut
\begin{lstlisting}[language=Python]
A[:, 0:4]
\end{lstlisting}
\strut
\end{minipage} & \begin{minipage}[t]{0.20\columnwidth}\raggedright\strut
\begin{lstlisting}
A[:, 1:4]
\end{lstlisting}
\strut
\end{minipage}\tabularnewline
\begin{minipage}[t]{0.23\columnwidth}\raggedright\strut
Remove a row\strut
\end{minipage} & \begin{minipage}[t]{0.23\columnwidth}\raggedright\strut
\begin{lstlisting}[language=Matlab]
A([1 2 4], :)
\end{lstlisting}
\strut
\end{minipage} & \begin{minipage}[t]{0.23\columnwidth}\raggedright\strut
\begin{lstlisting}[language=Python]
A[[0, 1, 3], :]
\end{lstlisting}
\strut
\end{minipage} & \begin{minipage}[t]{0.20\columnwidth}\raggedright\strut
\begin{lstlisting}
A[[1, 2, 4], :]
\end{lstlisting}
\strut
\end{minipage}\tabularnewline
\begin{minipage}[t]{0.23\columnwidth}\raggedright\strut
Diagonals of matrix\strut
\end{minipage} & \begin{minipage}[t]{0.23\columnwidth}\raggedright\strut
\begin{lstlisting}[language=Matlab]
diag(A)
\end{lstlisting}
\strut
\end{minipage} & \begin{minipage}[t]{0.23\columnwidth}\raggedright\strut
\begin{lstlisting}[language=Python]
np.diag(A)
\end{lstlisting}
\strut
\end{minipage} & \begin{minipage}[t]{0.20\columnwidth}\raggedright\strut
\begin{lstlisting}
diag(A)
\end{lstlisting}
\strut
\end{minipage}\tabularnewline
\begin{minipage}[t]{0.23\columnwidth}\raggedright\strut
Get dimensions of matrix\strut
\end{minipage} & \begin{minipage}[t]{0.23\columnwidth}\raggedright\strut
\begin{lstlisting}[language=Matlab]
[nrow ncol] = size(A)
\end{lstlisting}
\strut
\end{minipage} & \begin{minipage}[t]{0.23\columnwidth}\raggedright\strut
\begin{lstlisting}[language=Python]
nrow, ncol = np.shape(A)
\end{lstlisting}
\strut
\end{minipage} & \begin{minipage}[t]{0.20\columnwidth}\raggedright\strut
\begin{lstlisting}
nrow, ncol = size(A)
\end{lstlisting}
\strut
\end{minipage}\tabularnewline
\bottomrule
\end{longtable}

\section{Mathematical Operations}\label{mathematical-operations}

\begin{longtable}[]{@{}llll@{}}
\toprule
\begin{minipage}[b]{0.23\columnwidth}\raggedright\strut
Operation\strut
\end{minipage} & \begin{minipage}[b]{0.22\columnwidth}\raggedright\strut
MATLAB\strut
\end{minipage} & \begin{minipage}[b]{0.23\columnwidth}\raggedright\strut
Python\strut
\end{minipage} & \begin{minipage}[b]{0.20\columnwidth}\raggedright\strut
Julia\strut
\end{minipage}\tabularnewline
\midrule
\endhead
\begin{minipage}[t]{0.23\columnwidth}\raggedright\strut
Dot product\strut
\end{minipage} & \begin{minipage}[t]{0.22\columnwidth}\raggedright\strut
\begin{lstlisting}[language=Matlab]
dot(A, B)
\end{lstlisting}
\strut
\end{minipage} & \begin{minipage}[t]{0.23\columnwidth}\raggedright\strut
\begin{lstlisting}[language=Python]
np.dot(A, B) or A @ B
\end{lstlisting}
\strut
\end{minipage} & \begin{minipage}[t]{0.20\columnwidth}\raggedright\strut
\begin{lstlisting}
dot(A, B)
\end{lstlisting}
\strut
\end{minipage}\tabularnewline
\begin{minipage}[t]{0.23\columnwidth}\raggedright\strut
Matrix multiplication\strut
\end{minipage} & \begin{minipage}[t]{0.22\columnwidth}\raggedright\strut
\begin{lstlisting}[language=Matlab]
A * B
\end{lstlisting}
\strut
\end{minipage} & \begin{minipage}[t]{0.23\columnwidth}\raggedright\strut
\begin{lstlisting}[language=Python]
A @ B
\end{lstlisting}
\strut
\end{minipage} & \begin{minipage}[t]{0.20\columnwidth}\raggedright\strut
\begin{lstlisting}
A * B
\end{lstlisting}
\strut
\end{minipage}\tabularnewline
\begin{minipage}[t]{0.23\columnwidth}\raggedright\strut
Element-wise multiplication\strut
\end{minipage} & \begin{minipage}[t]{0.22\columnwidth}\raggedright\strut
\begin{lstlisting}[language=Matlab]
A .* B
\end{lstlisting}
\strut
\end{minipage} & \begin{minipage}[t]{0.23\columnwidth}\raggedright\strut
\begin{lstlisting}[language=Python]
A * B
\end{lstlisting}
\strut
\end{minipage} & \begin{minipage}[t]{0.20\columnwidth}\raggedright\strut
\begin{lstlisting}
A .* B
\end{lstlisting}
\strut
\end{minipage}\tabularnewline
\begin{minipage}[t]{0.23\columnwidth}\raggedright\strut
Matrix to a power\strut
\end{minipage} & \begin{minipage}[t]{0.22\columnwidth}\raggedright\strut
\begin{lstlisting}[language=Matlab]
A^2
\end{lstlisting}
\strut
\end{minipage} & \begin{minipage}[t]{0.23\columnwidth}\raggedright\strut
\begin{lstlisting}[language=Python]
np.linalg.matrix_power(A, 2)
\end{lstlisting}
\strut
\end{minipage} & \begin{minipage}[t]{0.20\columnwidth}\raggedright\strut
\begin{lstlisting}
A^2
\end{lstlisting}
\strut
\end{minipage}\tabularnewline
\begin{minipage}[t]{0.23\columnwidth}\raggedright\strut
Matrix to a power, elementwise\strut
\end{minipage} & \begin{minipage}[t]{0.22\columnwidth}\raggedright\strut
\begin{lstlisting}[language=Matlab]
A.^2
\end{lstlisting}
\strut
\end{minipage} & \begin{minipage}[t]{0.23\columnwidth}\raggedright\strut
\begin{lstlisting}[language=Python]
A**2
\end{lstlisting}
\strut
\end{minipage} & \begin{minipage}[t]{0.20\columnwidth}\raggedright\strut
\begin{lstlisting}
A.^2
\end{lstlisting}
\strut
\end{minipage}\tabularnewline
\begin{minipage}[t]{0.23\columnwidth}\raggedright\strut
Inverse\strut
\end{minipage} & \begin{minipage}[t]{0.22\columnwidth}\raggedright\strut
\begin{lstlisting}[language=Matlab]
inv(A)
\end{lstlisting}

or

\begin{lstlisting}[language=Matlab]
A^(-1)
\end{lstlisting}
\strut
\end{minipage} & \begin{minipage}[t]{0.23\columnwidth}\raggedright\strut
\begin{lstlisting}[language=Python]
np.linalg.inv(A)
\end{lstlisting}
\strut
\end{minipage} & \begin{minipage}[t]{0.20\columnwidth}\raggedright\strut
\begin{lstlisting}
inv(A)
\end{lstlisting}

or

\begin{lstlisting}
A^(-1)
\end{lstlisting}
\strut
\end{minipage}\tabularnewline
\begin{minipage}[t]{0.23\columnwidth}\raggedright\strut
Determinant\strut
\end{minipage} & \begin{minipage}[t]{0.22\columnwidth}\raggedright\strut
\begin{lstlisting}[language=Matlab]
det(A)
\end{lstlisting}
\strut
\end{minipage} & \begin{minipage}[t]{0.23\columnwidth}\raggedright\strut
\begin{lstlisting}[language=Python]
np.linalg.det(A)
\end{lstlisting}
\strut
\end{minipage} & \begin{minipage}[t]{0.20\columnwidth}\raggedright\strut
\begin{lstlisting}
det(A)
\end{lstlisting}
\strut
\end{minipage}\tabularnewline
\begin{minipage}[t]{0.23\columnwidth}\raggedright\strut
Eigenvalues and eigenvectors\strut
\end{minipage} & \begin{minipage}[t]{0.22\columnwidth}\raggedright\strut
\begin{lstlisting}[language=Matlab]
[vec, val] = eig(A)
\end{lstlisting}
\strut
\end{minipage} & \begin{minipage}[t]{0.23\columnwidth}\raggedright\strut
\begin{lstlisting}[language=Python]
val, vec = np.linalg.eig(A)
\end{lstlisting}
\strut
\end{minipage} & \begin{minipage}[t]{0.20\columnwidth}\raggedright\strut
\begin{lstlisting}
val, vec = eig(A)
\end{lstlisting}
\strut
\end{minipage}\tabularnewline
\begin{minipage}[t]{0.23\columnwidth}\raggedright\strut
Euclidean norm\strut
\end{minipage} & \begin{minipage}[t]{0.22\columnwidth}\raggedright\strut
\begin{lstlisting}[language=Matlab]
norm(A)
\end{lstlisting}
\strut
\end{minipage} & \begin{minipage}[t]{0.23\columnwidth}\raggedright\strut
\begin{lstlisting}[language=Python]
np.linalg.norm(A)
\end{lstlisting}
\strut
\end{minipage} & \begin{minipage}[t]{0.20\columnwidth}\raggedright\strut
\begin{lstlisting}
norm(A)
\end{lstlisting}
\strut
\end{minipage}\tabularnewline
\begin{minipage}[t]{0.23\columnwidth}\raggedright\strut
Solve linear system \(Ax=b\) (when \(A\) is square)\strut
\end{minipage} & \begin{minipage}[t]{0.22\columnwidth}\raggedright\strut
\begin{lstlisting}[language=Matlab]
A\b
\end{lstlisting}
\strut
\end{minipage} & \begin{minipage}[t]{0.23\columnwidth}\raggedright\strut
\begin{lstlisting}[language=Python]
np.linalg.solve(A, b)
\end{lstlisting}
\strut
\end{minipage} & \begin{minipage}[t]{0.20\columnwidth}\raggedright\strut
\begin{lstlisting}
A\b
\end{lstlisting}
\strut
\end{minipage}\tabularnewline
\begin{minipage}[t]{0.23\columnwidth}\raggedright\strut
Solve least squares problem \(Ax=b\) (when \(A\) is rectangular)\strut
\end{minipage} & \begin{minipage}[t]{0.22\columnwidth}\raggedright\strut
\begin{lstlisting}[language=Matlab]
A\b
\end{lstlisting}
\strut
\end{minipage} & \begin{minipage}[t]{0.23\columnwidth}\raggedright\strut
\begin{lstlisting}[language=Python]
np.linalg.lstsq(A, b)
\end{lstlisting}
\strut
\end{minipage} & \begin{minipage}[t]{0.20\columnwidth}\raggedright\strut
\begin{lstlisting}
A\b
\end{lstlisting}
\strut
\end{minipage}\tabularnewline
\bottomrule
\end{longtable}

\section{Sum / max / min}\label{sum-max-min}

\begin{longtable}[]{@{}llll@{}}
\toprule
\begin{minipage}[b]{0.23\columnwidth}\raggedright\strut
Operation\strut
\end{minipage} & \begin{minipage}[b]{0.22\columnwidth}\raggedright\strut
MATLAB\strut
\end{minipage} & \begin{minipage}[b]{0.24\columnwidth}\raggedright\strut
Python\strut
\end{minipage} & \begin{minipage}[b]{0.20\columnwidth}\raggedright\strut
Julia\strut
\end{minipage}\tabularnewline
\midrule
\endhead
\begin{minipage}[t]{0.23\columnwidth}\raggedright\strut
Sum / max / min of each column\strut
\end{minipage} & \begin{minipage}[t]{0.22\columnwidth}\raggedright\strut
\begin{lstlisting}[language=Matlab]
sum(A, 1)
max(A, [], 1)
min(A, [], 1)
\end{lstlisting}
\strut
\end{minipage} & \begin{minipage}[t]{0.24\columnwidth}\raggedright\strut
\begin{lstlisting}[language=Python]
sum(A, 0)
np.amax(A, 0)
np.amin(A, 0)
\end{lstlisting}
\strut
\end{minipage} & \begin{minipage}[t]{0.20\columnwidth}\raggedright\strut
\begin{lstlisting}
sum(A, 1)
maximum(A, 1)
minimum(A, 1)
\end{lstlisting}
\strut
\end{minipage}\tabularnewline
\begin{minipage}[t]{0.23\columnwidth}\raggedright\strut
Sum / max / min of each row\strut
\end{minipage} & \begin{minipage}[t]{0.22\columnwidth}\raggedright\strut
\begin{lstlisting}[language=Matlab]
sum(A, 2)
max(A, [], 2)
min(A, [], 2)
\end{lstlisting}
\strut
\end{minipage} & \begin{minipage}[t]{0.24\columnwidth}\raggedright\strut
\begin{lstlisting}[language=Python]
sum(A, 1)
np.amax(A, 1)
np.amin(A, 1)
\end{lstlisting}
\strut
\end{minipage} & \begin{minipage}[t]{0.20\columnwidth}\raggedright\strut
\begin{lstlisting}
sum(A, 2)
maximum(A, 2)
minimum(A, 2)
\end{lstlisting}
\strut
\end{minipage}\tabularnewline
\begin{minipage}[t]{0.23\columnwidth}\raggedright\strut
Sum / max / min of entire matrix\strut
\end{minipage} & \begin{minipage}[t]{0.22\columnwidth}\raggedright\strut
\begin{lstlisting}[language=Matlab]
sum(A(:))
max(A(:))
min(A(:))
\end{lstlisting}
\strut
\end{minipage} & \begin{minipage}[t]{0.24\columnwidth}\raggedright\strut
\begin{lstlisting}[language=Python]
np.sum(A)
np.amax(A)
np.amin(A)
\end{lstlisting}
\strut
\end{minipage} & \begin{minipage}[t]{0.20\columnwidth}\raggedright\strut
\begin{lstlisting}
sum(A)
maximum(A)
minimum(A)
\end{lstlisting}
\strut
\end{minipage}\tabularnewline
\begin{minipage}[t]{0.23\columnwidth}\raggedright\strut
Cumulative sum / max / min by row\strut
\end{minipage} & \begin{minipage}[t]{0.22\columnwidth}\raggedright\strut
\begin{lstlisting}[language=Matlab]
cumsum(A, 1)
cummax(A, 1)
cummin(A, 1)
\end{lstlisting}
\strut
\end{minipage} & \begin{minipage}[t]{0.24\columnwidth}\raggedright\strut
\begin{lstlisting}[language=Python]
np.cumsum(A, 0)
np.maximum.accumulate(A, 0)
np.minimum.accumulate(A, 0)
\end{lstlisting}
\strut
\end{minipage} & \begin{minipage}[t]{0.20\columnwidth}\raggedright\strut
\begin{lstlisting}
cumsum(A, 1)
cummax(A, 1)
cummin(A, 1)
\end{lstlisting}
\strut
\end{minipage}\tabularnewline
\begin{minipage}[t]{0.23\columnwidth}\raggedright\strut
Cumulative sum / max / min by column\strut
\end{minipage} & \begin{minipage}[t]{0.22\columnwidth}\raggedright\strut
\begin{lstlisting}[language=Matlab]
cumsum(A, 2)
cummax(A, 2)
cummin(A, 2)
\end{lstlisting}
\strut
\end{minipage} & \begin{minipage}[t]{0.24\columnwidth}\raggedright\strut
\begin{lstlisting}[language=Python]
np.cumsum(A, 1)
np.maximum.accumulate(A, 1)
np.minimum.accumulate(A, 1)
\end{lstlisting}
\strut
\end{minipage} & \begin{minipage}[t]{0.20\columnwidth}\raggedright\strut
\begin{lstlisting}
cumsum(A, 2)
cummax(A, 2)
cummin(A, 2)
\end{lstlisting}
\strut
\end{minipage}\tabularnewline
\bottomrule
\end{longtable}

\section{Programming}\label{programming}

\begin{longtable}[]{@{}llll@{}}
\toprule
\begin{minipage}[b]{0.19\columnwidth}\raggedright\strut
Operation\strut
\end{minipage} & \begin{minipage}[b]{0.22\columnwidth}\raggedright\strut
MATLAB\strut
\end{minipage} & \begin{minipage}[b]{0.22\columnwidth}\raggedright\strut
Python\strut
\end{minipage} & \begin{minipage}[b]{0.25\columnwidth}\raggedright\strut
Julia\strut
\end{minipage}\tabularnewline
\midrule
\endhead
\begin{minipage}[t]{0.19\columnwidth}\raggedright\strut
Comment one line\strut
\end{minipage} & \begin{minipage}[t]{0.22\columnwidth}\raggedright\strut
\begin{lstlisting}[language=Matlab]
% This is a comment
\end{lstlisting}
\strut
\end{minipage} & \begin{minipage}[t]{0.22\columnwidth}\raggedright\strut
\begin{lstlisting}[language=Python]
# This is a comment
\end{lstlisting}
\strut
\end{minipage} & \begin{minipage}[t]{0.25\columnwidth}\raggedright\strut
\begin{lstlisting}
# This is a comment
\end{lstlisting}
\strut
\end{minipage}\tabularnewline
\begin{minipage}[t]{0.19\columnwidth}\raggedright\strut
Comment block\strut
\end{minipage} & \begin{minipage}[t]{0.22\columnwidth}\raggedright\strut
\begin{lstlisting}[language=Matlab]
%{
Comment block
%}
\end{lstlisting}
\strut
\end{minipage} & \begin{minipage}[t]{0.22\columnwidth}\raggedright\strut
\begin{lstlisting}[language=Python]
# Block
# comment
# following PEP8
\end{lstlisting}
\strut
\end{minipage} & \begin{minipage}[t]{0.25\columnwidth}\raggedright\strut
\begin{lstlisting}
#=
Comment block
=#
\end{lstlisting}
\strut
\end{minipage}\tabularnewline
\begin{minipage}[t]{0.19\columnwidth}\raggedright\strut
For loop\strut
\end{minipage} & \begin{minipage}[t]{0.22\columnwidth}\raggedright\strut
\begin{lstlisting}[language=Matlab]
for i = 1:N
   % do something
end
\end{lstlisting}
\strut
\end{minipage} & \begin{minipage}[t]{0.22\columnwidth}\raggedright\strut
\begin{lstlisting}[language=Python]
for i in range(n):
    # do something
\end{lstlisting}
\strut
\end{minipage} & \begin{minipage}[t]{0.25\columnwidth}\raggedright\strut
\begin{lstlisting}
for i = 1:N
   # do something
end
\end{lstlisting}
\strut
\end{minipage}\tabularnewline
\begin{minipage}[t]{0.19\columnwidth}\raggedright\strut
While loop\strut
\end{minipage} & \begin{minipage}[t]{0.22\columnwidth}\raggedright\strut
\begin{lstlisting}[language=Matlab]
while i <= N
   % do something
end
\end{lstlisting}
\strut
\end{minipage} & \begin{minipage}[t]{0.22\columnwidth}\raggedright\strut
\begin{lstlisting}[language=Python]
while i <= N:
    # do something
\end{lstlisting}
\strut
\end{minipage} & \begin{minipage}[t]{0.25\columnwidth}\raggedright\strut
\begin{lstlisting}
while i <= N
   # do something
end
\end{lstlisting}
\strut
\end{minipage}\tabularnewline
\begin{minipage}[t]{0.19\columnwidth}\raggedright\strut
If\strut
\end{minipage} & \begin{minipage}[t]{0.22\columnwidth}\raggedright\strut
\begin{lstlisting}[language=Matlab]
if i <= N
   % do something
end
\end{lstlisting}
\strut
\end{minipage} & \begin{minipage}[t]{0.22\columnwidth}\raggedright\strut
\begin{lstlisting}[language=Python]
if i <= N:
   # do something
\end{lstlisting}
\strut
\end{minipage} & \begin{minipage}[t]{0.25\columnwidth}\raggedright\strut
\begin{lstlisting}
if i <= N
   # do something
end
\end{lstlisting}
\strut
\end{minipage}\tabularnewline
\begin{minipage}[t]{0.19\columnwidth}\raggedright\strut
If / else\strut
\end{minipage} & \begin{minipage}[t]{0.22\columnwidth}\raggedright\strut
\begin{lstlisting}[language=Matlab]
if i <= N
   % do something
else
   % do something else
end
\end{lstlisting}
\strut
\end{minipage} & \begin{minipage}[t]{0.22\columnwidth}\raggedright\strut
\begin{lstlisting}[language=Python]
if i <= N:
    # do something
else:
    # so something else
\end{lstlisting}
\strut
\end{minipage} & \begin{minipage}[t]{0.25\columnwidth}\raggedright\strut
\begin{lstlisting}
if i <= N
   # do something
else
   # do something else
end
\end{lstlisting}
\strut
\end{minipage}\tabularnewline
\begin{minipage}[t]{0.19\columnwidth}\raggedright\strut
Print text and variable\strut
\end{minipage} & \begin{minipage}[t]{0.22\columnwidth}\raggedright\strut
\begin{lstlisting}[language=Matlab]
x = 10
fprintf('x = %d \n', x)
\end{lstlisting}
\strut
\end{minipage} & \begin{minipage}[t]{0.22\columnwidth}\raggedright\strut
\begin{lstlisting}[language=Python]
x = 10
print('x = {}'.format(x))
\end{lstlisting}
\strut
\end{minipage} & \begin{minipage}[t]{0.25\columnwidth}\raggedright\strut
\begin{lstlisting}
x = 10
println("x = $x")
\end{lstlisting}
\strut
\end{minipage}\tabularnewline
\begin{minipage}[t]{0.19\columnwidth}\raggedright\strut
Function: one line/ anonymous\strut
\end{minipage} & \begin{minipage}[t]{0.22\columnwidth}\raggedright\strut
\begin{lstlisting}[language=Matlab]
f = @(x) x^2
\end{lstlisting}
\strut
\end{minipage} & \begin{minipage}[t]{0.22\columnwidth}\raggedright\strut
\begin{lstlisting}[language=Python]
f = lambda x: x**2
\end{lstlisting}
\strut
\end{minipage} & \begin{minipage}[t]{0.25\columnwidth}\raggedright\strut
\begin{lstlisting}
f(x) = x^2
\end{lstlisting}
\strut
\end{minipage}\tabularnewline
\begin{minipage}[t]{0.19\columnwidth}\raggedright\strut
Function: multiple lines\strut
\end{minipage} & \begin{minipage}[t]{0.22\columnwidth}\raggedright\strut
\begin{lstlisting}[language=Matlab]
function out  = f(x)
   out = x^2
end
\end{lstlisting}
\strut
\end{minipage} & \begin{minipage}[t]{0.22\columnwidth}\raggedright\strut
\begin{lstlisting}[language=Python]
def f(x):
    return x**2
\end{lstlisting}
\strut
\end{minipage} & \begin{minipage}[t]{0.25\columnwidth}\raggedright\strut
\begin{lstlisting}
function f(x)
   return x^2
end
\end{lstlisting}
\strut
\end{minipage}\tabularnewline
\bottomrule
\end{longtable}

In the Python code we assume that you have already run
\lstinline!import numpy as np!

\end{document}
